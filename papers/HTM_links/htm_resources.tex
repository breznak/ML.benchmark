\documentclass[a4,IEEEconf]{article}
\usepackage{graphicx}
\usepackage{bibcheck}
%!\usepackage{spelling}


%opening
\title{Hierarchical Temporal Memory - literature research \& community ecosystem}
\author{Marek Otahal, Olga Stepankova}
\makeindex

\begin{document}

\maketitle

\begin{abstract}
This is a working DRAFT, although comments, corrections and contributions are very welcome! 

The idea is to cover all available \textit{literature} about Hierarchical Temporal Memory \(HTM\) and offer an overview of the community \textit{ecosystem}: focus-specific projects, support tools for HTM, alternative implementations, etc. 

The text would be divided into logical topics, each providing a brief description and references to the literature in Bibliography.
\end{abstract}

% TOC
\tableofcontents

% Intro
\section{Introduction}
TODO Outline and explanation of this document.
\subsection{HTM}
Short into to HTM
\subsection{Community ecosystem}
types of resources: numenta, ML, videos, meetups, hackathons, projects, ...

% HTM
\section{HTM Theory}
\subsection{Hierarchy}
\subsection{Sparse, distributed representation}
Sparse, distributed, semantic vectors, ...
\subsection{Biological theory behind HTM}
cortex, columnar structure, synapses, ...
Not so deep, as this will be covered in a separate paper.
\subsection{Predictions and Anomaly detection}
main HTM functionality, briefly compare, explain
\subsection{Discussion}
problems and ideas of this section

\section{Implementations}
\subsection{NuPIC}
"Main" implementation
\subsection{Language ports}
Java, C++, ...?
\subsection{Specialized functionality}
Continuous, task-specific \(nupic.vision, nupic.nlp,...\), biological, ...
\subsection{Discussion}
Speed issues, simplified codebase, ...

\section{Ecosystem}
The community ecosystem, resources, projects and activities. 
\subsection{Resources}
numenta.org, ML, github, gitter, videos, hackathons \& meetups, ...
\subsection{Sensory processing}
vision, audio, NLP, ...
\subsection{Applications}
apps of nupic
\subsection{Visualizations \& IDEs}
tools to help visualize and debug HTMs
\subsection{Support}
Connectors HTM2..., ??
\subsection{Research}
NAB, ML.benchmarks, vision, ...

\section{Interested parties}
3rd party subjects that are using HTM, or could be interested to do so
\subsection{Using NuPIC}
Grok, ...
\subsection{Could be used with HTM}
cortical.IO, ...
\subsection{Interested}
Areas where HTM has been,or could be applied. 

\section{Discussion}
Overall comments and thoughts

\section{Conclusion}
brief summary

\end{document}
