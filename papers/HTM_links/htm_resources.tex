\documentclass[a4,IEEEconf]{article}
\usepackage{graphicx}
\usepackage{bibcheck}
%!\usepackage{spelling}
% Load the package with the acronym option
\usepackage[acronym,footnote]{glossaries} 

  
% % Acronym definitions
\newacronym{htm}{HTM}{Hierarchical Temporal Memory}
\newacronym{nupic}{NuPIC}{Numenta Platform for Intelligent Computing}
\newacronym{cla}{CLA}{Cortical Learning Algorithm}
\newacronym{sdr}{SDR}{Sparse Distributed Representation}

% Generate the glossary
\makeglossaries


%opening
\title{Hierarchical Temporal Memory - literature research \& community ecosystem}
\author{Marek Otahal, Olga Stepankova}
\makeindex

\begin{document}

\maketitle
\bibliographystyle{alpha}


\begin{abstract}
This is a working DRAFT, although comments, corrections and contributions are very welcome! 

The idea is to cover all available \textit{literature} about \gls{htm} and offer an overview of the community \textit{ecosystem}: focus-specific projects, support tools for HTM, alternative implementations, etc. 

The text would be divided into logical topics, each providing a brief description and references to the literature in Bibliography.
\end{abstract}

% TOC
\tableofcontents
% acronyms
\printglossaries

% Intro
\section{Introduction}
Aim of this paper is to present a sorted summary of resources about \gls{htm}, its alternative implementations and projects around it. 

The community and number of materials is rather vast, so we believe this overview will be useful for a potential new users, looking to make a rough idea what is \gls{htm}, what can they achieve with it, etc. 

Other researchers will find useful the commented bibliography organized in topics of interest. 

First part of the document covers the theory behind \gls{htm} and highlights the most important materials for the core \textit{principles}. The next section focuses on \textit{implementations} in different programming languages and their status. This will also introduce alternative implementations, aiming for a specific functionality or tasks. In the section about \textit{the community ecosystem} we describe projects with a wider focus, including various visualization and debugging tools, resources other than scientific papers - which are common for \gls{nupic} - videos, hackathons, meet-ups, \dots This section mentions applications of \gls{htm} in different cognitive areas and also practical applications in the industry. 

The main contribution of this paper is providing the reader a comprehensive collection of literature research on HTM, structured into specific topics. It also serves as an overview to the numerous implementations, developers and projects gathering around NuPIC community, which will be useful for seeking potential partners in applications of HTM. 

% HTM
\section{HTM Theory}
The section focus is on \gls{htm} theory itself. We will briefly introduce the origins of the theory, its main principles - focusing not only on the articles for \gls{htm} but we provide references to similar topics in other theories as well. For \gls{htm}, the most improtant concept is the \textit{biological plausibility}, while for many other (deeplearning) neural network researchers is the \textit{mathematical formalism} paramount. Research papers on \textit{anomaly detection} and continuous prediction are presented, as this is the most common functionality of the models.  
\subsection{History}
The first publication on HTM dates to the year 2004 when the author, Jeff Hawkins, introduces his theory about brain functionality (precisely neocortex) in the book "On Intelligence" \cite{hawkins04}. On a high level description the book introduces the ideas of importance of hierarchies in brain, how everything in the cortex is processed as a stream of sequential data and how online learning and anomaly detection could be performed by the cortical regions. 

First more detailed description of concepts how spatial and temporal memory is implemented in the cortex came in \cite{HawkinsAndGeorge2006}. The implementation of this system was commercially developed by the Numenta.org TODO-link company. This system focused heavily on hierarchies and at the time has been used as a state of the art method for real-time object tracking in video. 

An important update released in 2010, with introduction of \gls{cla} \cite{Hawkins:2010}, a biologically plausible learning algorithm for the HTM. 
Another important step in popularization of HTM has been that the software and patents have been released that year for public use as an open-source software, organized around NuPIC community. 

In the recent year, a broader community of researchers started to focus on HTM, mainly on the neuroscientific qualities of the system, for example \cite{journals/corr/Ferrier14},\cite{journals/corr/Byrne15a}, \cite{journals/corr/HawkinsA15}. 

A study explaining the representation of information in the brain, or at least in the HTM theory came out in 2015 \cite{journals/corr/AhmadH15} and evaluation of HTM on the anomaly detection task on streaming data published in \cite{journals/corr/LavinA15}. The latest publication as of today is from early 2016 on mathematical formalism of HTM \cite{1601.06116v1}. 

\subsection{Principles}
HTM is often compared with nowadays popular \textit{deep learning} approaches, while both successfully developed in the recent years, the deeplearning methods are built bottom-up and offer mathematical proofs of correctness, complexity, etc.  (TODO say it like that?) The deep-learning techniques take advantage from truly large scale networks' computational power enabled only recently with the modern hardware. HTM, on the other hand, focuses on biological correctness and will adhere to the theories and processes discovered from neurology of the mammalian neocortex. HTM is built top-down and the integral parts (neurons) and processes are much more complicated. As Yann Le-Cunn stated TODO. 

Even though the exact details are still being refined by progress in neuroscience, the core principles of HTM have been described in the first publication \cite{hawkins04} and can be summarized to: focus on biological plausibility, the unifying principle of the neocortex, abstraction through hierarchies and processing of continuous streams of information. 

On a more technical details, the main properties include: SDRs as representation of information in the brain, spatio-temporal pooling and synaptic adaptation. 

\subsubsection{Basic materials}
HTM is a \textit{neural network} model, biologically derived from the principles of mammalian \textit{neocortex} which performs unsupervised online learning on \textit{streaming} data. This, a specific learning mechanism called CLA, and the fact that the terminology is quite non-standard (in terms of computer science (CS), as it's mostly derived from neuroscience) causes HTM to have a rather steep learning curve, even for experts from "classical" neural networks and machine learning. 

The introductory publication is the book "On Intelligence" \cite{hawkins04}, more technical details and principles about the SDRs and spatio-temporal pooling are explained in the "HTM Whitepaper" \cite{Hawkins:2010}. A specific folklore is that a number of (high quality and detailed) talks about HTM is published in a form of video materials, as lectures "talk with an expert on...", available on Numenta's youtube channel TODO-ref. The public repository with the implementation and accompanied documentation is a great place for further investigation for a newcomer. The community is very active on mailing lists (ML) focused on programming implementation, theory and general questions. 
  
\subsubsection{Unifying principle}
Briefly, the core observation is that the mammalian \cite{Karten97} cortex \cite{Marr70}\cite{Elston03} performs many high-level cognitive functions \cite{hawkins04}\cite{Hawkins:2010}\cite{journals/corr/Ferrier14}\cite{journals/corr/Byrne15a}\cite{journals/ijon/Lo12} (sensory processing: vision\cite{tso:connectivity}, hearing/speech/language processing (NLP)\cite{Zatorre2002}, motor\cite{RizzolattiFogassiGallese02} operations; novelty (anomaly) detection, etc.), while these are very varied tasks, the studies suggest the structure \cite{DouglasMartin04} of the neocortex \cite{Bienenstock95} is relatively uniform \cite{chaterv03},\cite{1511.00411v1},\cite{krubitzer:tins95} (structured into (micro)columns \cite{pmid9153131} and horizontal layers, with pyramidal neurons \cite{pmid19661433}  inside) \cite{Jones00}\cite{vondermalsburg:principles}! From the materialistic principle this should imply a unified learning algorithm\cite{OReillyRudy00},\cite{Hawkins:2010},\cite{Creutzfeldt77} ("theory of learning") exists. It is the objective of HTM to reconstruct such algorithm - the current proposed version is called the CLA. 


\subsubsection{Hierarchy}
Hierarchical organization is observed in the cortical regions, which are formed by horizontal layers \cite{pmid22147913} (for example in the visual cortical regions V1-V5 \cite{journals/pami/KrugerJKLLPRW13}, and simulated by the \textit{Blue Brain Project}\cite{pmid26451478}). The principle behind hierarchy is emergence of more stable patterns on higher levels (when the predictions are correct), this is a core property of deep learning \cite{Bengio09ftml}, \cite{journals/pami/SalakhutdinovTT13}. Idea is that feature extraction (and selection) is not that important, the useful features would emerge in the middle layers (for example representation of edges in vision tasks), on consecutive higher layers a more stable and abstract patterns form (eg. representation of certain objects, etc.) and the highest layer typically only performs classification if an observed feature has been detected or not. The same process with abstraction and stability is valid for HTM, while there are some differences, as the SDR representations used in HTM convey a fuzzy representation for multiple parallel concepts (in sequence memory), as explained later. 

\subsubsection{Sparse, distributed representation}
Sparse distributed representation (SDR) \cite{Hinton:97} is the core property of the HTM \cite{journals/corr/AhmadH15}, and is a plausible hypothesis for how the information is represented and transfered around in the brain \cite{pmid26749189}, \cite{1508.04554v1}, \cite{CansinoMaquetDolanEtAl02}. It is an extension to \textit{semantic embeddings}\cite{liu-semantic-1997}, \cite{turney2010frequency}, \cite{journals/corr/abs-1105-2868} in a way that semantically similar objects have similar representations, with low distance between them. What is extended is that the semantic representation in SDR use sparse coding \cite{pmid21315595} and form a distributed memory. Another distinct feature is that the weights are binary only \cite{cond-mat/9604102v1}, which allows for some nice semantic arithmetics. The representations can be subsampled and still convey the same meaning, or distracted ("minus", Like the well known NLP example from semantic vectors: "Apple -fruit means computer brand", which was popularised by Google's \textit{word2vec}), or added (union, "plus") to produce a more abstract term governing all the sub-representations. SDRs are highly robust to noise, in the HTM are produced by Spatial Pooler, or spatial memory. 

\subsubsection{Spatial pooling}
Spatial pooling \cite{Hinton:97},\cite{BeckerHinton93} is a statistical mechanism for grouping of features that accounts for topology and locality of the data, for such is most often used in vision problems \cite{journals/corr/HeZR014}. It is commonly used in deeplearning and also in the brain, where responsible background mechanism is (local) inhibition \cite{devalois:inhibition}. Spatial pooling is in HTM performed by so called \textit{SpatialPooler} module and is responsible for transforming the (binary) input vector  to the sparse distributed representation (SDR).\cite{journals/corr/AhmadH15} 

\subsubsection{Temporal memory}
HTM is performing unsupervised online learning \cite{1504.02518v2} (commonly applied to videos \cite{journals/corr/PigouODHD15}) with its statistical temporal generative model. In HTM this is achieved with a \textit{temporal pooling} mechanism \cite{journals/ijon/HurriH03}, \cite{journals/corr/WangCSLS15}, \cite{SutskeverETAL:09} that utilizes (parallel) context tracking of sequential data (time-series) and continuously provides predictions and anomaly scores. Many contexts are kept simultaneously with cells within columns and proximal dendrite connections \cite{Hawkins:2010}.
  
\subsubsection{Anomaly detection}
Online \textit{anomaly detection} \cite{journals/tissec/LaneB99} on streaming (temporal, sequential) data is relatively popular task (with example use-cases in many industrial domains \cite{conf/bigdataconf/RettigKCP15}, including: IT server load monitoring, financial predictions, security, sensory measurements \cite{journals/corr/abs-0810-5157}, etc.), commonly coupled with continuous monitoring and visualization. For HTM providing an instant anomaly score is an implicit property of the model. 
The task has been benchmarked \cite{journals/corr/LavinA15} on real-world (annotated) datasets and compared with other suitable machine-learning (ML) models. 
 
\subsection{Biological background}
HTM belongs to the group of biologically inspired (more precisely cortical) neural network models, which includes connectomes \cite{journals/scholarpedia/Sporns10},namely the Blue Brain Project \cite{markram_blue_2006} - aiming to exactly model the brain functions to the lowest possible level. This makes these models very useful for neuro-scientific, psychological or medical research. Another category is modeling only certain structures or functions of the brain circuitry (cortex, thalamus, visual cortex, etc. \cite{ccortex99-13}). The last group are \textit{bio-inspired} models that strive for both biological accuracy and practical usefulness in applications other than brain-circuitry simulation. That is bound by computational limits of the current hardware so these approaches must conform to some level of simplification in order to optimize their execution. HTM belongs to this group along with \textit{deep neural networks} \cite{1601.06116v1},  \textit{self-organizing maps (SOM)} \cite{journals/ni/BednarKM04} and \textit{spiking neural networks} \cite{Izhikevich2004} and generally nonlinear brain dynamics models \cite{q-bio/0507014v1}. 

The recent research in HTM \cite{conf/ijcnn/KnellerT15}, \cite{journals/jips/Kang15}, \cite{journals/corr/Ferrier14} and \cite{journals/corr/Byrne15a}. 

\subsection{Mathematical formalization}
As mentioned, HTM is taking direction of \textit{biological plausibility} and unlike deeplearning \cite{journals/nn/Schmidhuber15} (deep belief networks, ...) it does not have a fully developed mathematical theory with exact solutions \cite{journals/corr/SaxeMG13} behind it. The mathematical background of HTM has been published in the original publication \cite{HawkinsAndGeorge2006}, further refined for cortical-circuitry \cite{pmid19816557}. The most recent publications cover mathematics behinds sparse distributed representations (robustness, capacity, etc.) \cite{journals/corr/AhmadH15} and functionality of spatial pooling \cite{1601.06116v1}. 

\subsection{Discussion}
This section described HTM in context of other popular machine learning approaches (mainly (deep) neural networks), highlighted its focus on biological plausibility and its underlying principles (SDRs, hierarchy, universal algorithm). Mathematical description of the theory was often considered lacking, but it is improving recently and encourages confrontation and cooperation with wider research community. 

\section{Implementations}
The following section targets existing implementation of HTM - in different programming languages, with some extended (implementational) capabilities (eg. HTM in the cloud, or GPGPU accelerated), specialized functionality (HTM for vision) or building completely on a derived HTM theory. 

A brief description, status, feature parity, activity and main differences are noted for each of the implementations. 
\subsection{NuPIC}
The "main" implementation of HTM as descibed in \cite{Hawkins:2010} is called "NuPIC". It was originally developed by Numenta and open-sourced in 2010, since then serving as a reference implementation maintained in TODO. Its feature parity with the theory is almost 100 \% (even though the theory is also under development, e.g. in progress motor operations, or recently updated temporal memory). 

The implementation is in Python (2.7, with support for Python 3.x underway), there are unit \& regression tests and new features are tracked in the projects issue tracker. It supports serialization of the models with CAPnP and utilizes Numpy for vector algebra. 

The code base also spans additional utilities, such as convenience methods for running HTM models; OPF format for creating hierarchies and linking regions; parameter optimization framework with metrics; performance profiles, etc. - making it the most useful and convenient implementation. NuPIC can use "Swig" bindings for interfacing the core code written in another supported (by Swig) languages. 

Advantages come from the choice of the programming language, so it's elegant and efficient prototyping in Python, disadvantage would be speed limitation of the Python interpreter (due to Numpy, Swig and other libs faster interpreters such as Pypy are not supported, Cython unofficially works), and its focus on strict backwards compatibility, making larger modifications or refactoring very difficult. 

\subsection{Language ports}
There are several implementations of HTM (ports) in another languages. Their motivation is mostly for performance reasons (C++), for the "quality" of code (Java version is more object oriented), or just for the sake of reimplementation and availability in the target language. 

NuPIC (python) and its unit and regression tests are considered reference for any derived implementation. It should be noted these ported implementations, typically community-maintained, have varied feature parity with the official or can fall behind if left unmaintained and new changes get pushed to the NuPIC codebase. 

\subsubsection{C++}
The first and official implementation has been started for performance reasons and is developed in C++(11) and is called "nupic.core" TODO-ref. It has a status of an "official implementation" (which means it is maintained by Numenta), and the "core" functionality has been ported, it is actively developed and maintained by Numenta. 

The "core" means SpatialPooler, TemporalPooler (and TemporalMemory), Classifier, and OPF (Model, Region) classes.
What is missing from the HTM core are: Encoders (input layer) and Anomaly (output layer for anomaly detection models); nor functionality for parameter optimizations is implemented.  

Advantage is improved speed, but neither GPGPU support, multi-threaded parallel operation, nor speed-optimized algebraic libraries (Eigen) are supported, so the speed is not on par with methods like GPGPU-based deeplearning, which is clearly taking an edge at the brute-force processing. 

\subsubsection{Java}
The Java implementation called "htm.java" TODO-ref maintained by David Ray (@cogmission) is a first community based port to gain the official status. It is actively maintained and developed, feature parity is almost 100\% with NuPIC. 

Its advantage is that the codebase was rewritten from scratch and in OOP, so the code could be considered cleaner. 

\subsubsection{JavaScript}
A community port to JavaScript TODO-ref, currently unmaintained but has reached parity with core, so only encoders and anomaly code are missing for the full HTM stack. 

\subsection{Extended functionality}
Ports that somewhat extend the functionality of HTM, this includes task specific repositories (for sensory inputs) that usually provide additional encoders or preset models and experiment configurations. 

\subsubsection{Research}
"Nupic.research" is an official experimental repository TODO-ref, maintained by Numenta, actively developed but in very experimental state that may change rapidly. It is stated as a "platform for open research". It does not contain a full implementation of HTM per se, but contains a number of experiments, some of which change the integral parts of HTM (in implementation). For example the temporal memory (successor of temporal pooling) has been developed there, current are motor operation experiments, NLP, or union pooling. Verified changes from this repo are later transfered to the NuPIC official implementation. 

\subsubsection{Specialized task}
Task specific TODO
\subsection{Modifications to HTM}
In this category are implementations of modified HTM theories, these offer some radical change, could be very interesting to experiment with, but may be suitable only for a specific task, or need to be carefully validated. 
\subsubsection{Continuous CLA}
TODO
PaCLA, HSLT (Erik), ...

\subsection{Discussion}
To conclude this section about HTM, we should repeat that the philosophy of HTM is trying to achieve a unified learning algorithm suitable for all kinds of high-level cognitive tasks that the cortex is responsible for. For that it is built on emulating structure and processes found in the neocortex, 

That is a distincion from more mathematical models (deep neural networks, etc.) that compose of much simpler neurons and have a solid mathematical proofs for all their functionality. The mathematical theory is being finished for HTM and its sparse representations very recently. 

Another lacking functionality, with regards to the implementation, in HTM is the lack of speed optimizations. The NuPIC implementations lack support for GPGPU, or even ability to take advantage of multi-core systems, or optimized algebra libraries. Surprisingly, the algorithm is very apt even with its limitations and amounts of data it is able to precess. It will be interesting to see when the full potential is reached in this direction. 

There are several alternative implementations of HTM, either ports to another languages, or solutions focused on a specific functionality or otherwise modifying the CLA learning algorithm. This is a great things that the ports exist, what we should note (and is known for any large development project) is that the alternatives demand a huge amount of human resources, require work to keep them up to date with the main NuPIC and propagating any changes or improvements done in a fork to the main implementation or other alternatives is a complicated job. This would be true for any decentralized projects. 

In the following section I would like to highlight implementations that would be useful to have, but are not created yet. 
\subsubsection{Planned}
I believe an important step would be a \textit{minimalistic} implementation of the official Python library NuPIC. The "nupic.simple" should focus only on implementation of the core HTM concepts - the encoders for input layer, spatial pooling, temporal memory and anomaly detection (and classifier), that is all the HTM theory describes. No additional functionality, minimum (none) dependency libraries and no optimizations - it should represent the pseudo-code as closely as possible. Such code base could be used for bootstrapping any new ports or alternative implementations, or serving as a reference for new users who wish to understand just the core principles. 

An extension from the idea above would be "nupic.python", a \textit{Python-only} implementation of the whole NuPIC functionality. 

An interesting project would be a "nupic.clean" repository, implementation focused on an \textit{elegant code}, utilizing OOP, inheritance and other programming concepts. Some more radical refactoring would help NuPIC becoming more popular for new developers and increase the level of modularization, where new ideas can be tested in parallel with the main development with lesser burden of merging the changes. 

A specialized repository "nupic.bio" focusing on even higher \textit{biological accuracy and detail} would be useful for interdisciplinary experiments and simulations in biology, psychology and cognitive sciences. This is a place where HTM could be used more often. 

\section{Ecosystem}
The community ecosystem, resources, projects and activities. 
\subsection{Resources}
numenta.org, ML, github, gitter, videos, hackathons \& meetups, ...
\subsection{Sensory processing}
vision, audio, NLP, ...
\subsection{Applications}
apps of nupic
\subsection{Visualizations \& IDEs}
tools to help visualize and debug HTMs
\subsection{Support}
Connectors HTM2..., ??
\subsection{Research}
NAB, ML.benchmarks, vision, ...
\subsection{Interested parties}
3rd party subjects that are using HTM, or could be interested to do so
\subsubsection{Using NuPIC}
Grok, ...
\subsubsection{Could be used with HTM}
cortical.IO, ...

\section{Discussion}
Overall comments and thoughts
\subsection{Evaluation \& comparisons}
NAB, benchmark
\subsection{Interested}
Areas where HTM has been,or could be applied. 

\section{Conclusion}
brief summary

\section{Acknowledgement}

\section{Appendix A: HTM resources by Numenta}
This section collects references to scientific paterials published (solely) by Numenta, the "official" original HTM developing organization. 
TODO
\section{Appendix B: List of abbreviations}
TODO

\bibliography{htm_resources}

\end{document}
